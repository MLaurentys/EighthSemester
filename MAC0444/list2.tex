\documentclass[]{article}
\usepackage{amsmath}
\usepackage{bm}
\usepackage{amssymb}

\title{\vspace{-4.0cm}MAC0444 - List 2}
\author{Matheus T. de Laurentys, 9793714}
\begin{document}
	\maketitle
	\noindent
	\textbf{Q 1:} Dados os seguintes predicados:\\
	$fezEx(x) \rightarrow vaiBem(x)$\\
	$vaiBem(x) \rightarrow mediaAlta(x)$\\
	$mediaAlta(x) \rightarrow aprovado(x)$\\
	$fezEx(Joao)$\\
	$vaiBem(Maria)$\\
	Primeiro passá-los para cláusulas de Horn: \\
	\begin{enumerate}
		\item $\lnot fezEx(x) \lor vaiBem(x)$
		\item $\lnot vaiBem(x) \lor mediaAlta(x)$
		\item $\lnot mediaAlta(x) \lor aprovado(x)$
		\item $fezEx(Joao)$
		\item $vaiBem(Maria)$\\
	\end{enumerate}
	Vou adicionar a cláusula:\\
	$\cdots$ 6. $\lnot aprovado(Joao) \lor \lnot aprovado(Maria)$\\\\
	Todas as cláusula são de Horn. Podemos resolver usando SLD:\\\\
	\begin{tabular} {ll}
	(substituindo $x$ por João em 3:) &7. (6 + 3) $\lnot mediaAlta(Joao) \lor \lnot aprovado(Maria)$\\
	(substituindo $x$ por Maria em 3:) &8. (7 + 3) $\lnot mediaAlta(Joao) \lor \lnot mediaAlta(Maria)$\\
	(substituindo $x$ por João em 2:) &9. (8 + 2) $\lnot vaiBem(Joao) \lor \lnot mediaAlta(Maria)$\\
	(substituindo $x$ por Maria em 2:) &10. (9 + 2) $\lnot vaiBem(Joao) \lor \lnot vaiBem(Maria)$\\
	&11. (10 + 5) $\lnot vaiBem(Joao)$\\
	(substituindo $x$ por João em 1:) &12. (11 + 1) $\lnot fezEx(Joao)$\\
	&13. (12 + 4) false\\\\
	\end{tabular}
	Mostrado que ambos João e Maria foram aprovados.\\\\
	
\end{document}